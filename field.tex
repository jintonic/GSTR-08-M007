\subsection{Calculation of the Electric and Weighting Fields} 
\label{sec:field}
To calculate the electric field the poisson equation 
\begin{equation}
\nabla \varphi(\textbf{r}) = -\frac{1}{\epsilon_{0} \epsilon_{R}} \rho(\textbf{r})
\label{eq:poisson}
\end{equation} 
needs to be solved. Since coaxial detectors are used
the equation is rewritten in cylindrical coordinates 
\begin{equation}
\frac{1}{r} \frac{\partial \varphi}{\partial r} + \frac{\partial^{2} \varphi}{\partial r^{2}} + \frac{1}{r^{2}} \frac{\partial^{2} \varphi}{\partial \phi^{2}} +
\frac{\partial^{2} \varphi}{\partial z^{2}} = - \frac{1}{\epsilon_{0}
\epsilon_{R}} \rho(r).
\label{eq:poisson_cyl}
\end{equation}

According to the Shockley-Ramo theorem the weighting potential
$\varphi_{0}$ is needed to calculate the induced charges Q on the
electrodes. Therefore the voltage on the electrodes for which the
induced charge should be calculated is set to 1 and all others are set
to 0. No space charges are taken into account, i.e. the Laplace equation
needs to be solved. In cylindric coordinates this becomes 
\begin{equation}
\frac{1}{r}
\frac{\partial \varphi}{\partial r} + \frac{\partial^{2} \varphi}{\partial
r^{2}} + \frac{1}{r^{2}} \frac{\partial^{2} \varphi}{\partial \phi^{2}} +
\frac{\partial^{2} \varphi}{\partial z^{2}} = 0.
\label{eq:laplace}
\end{equation}

The electric field and the weighting potential are calculated on a
grid using a numerical technique called Successive Overrelaxation
(SOR). Numerical derivatives can be written as 
\begin{eqnarray}
\frac{\partial \varphi}{\partial x} = \frac{\varphi(x+1) - \varphi(x-1)}{2h_{x}} \label{eq:1stDerivative} \\
\frac{\partial^{2} \varphi}{\partial x^{2}} = \frac{\varphi(x+1) - 2 \varphi(x) + \varphi(x-1)}{h^{2}_{x}} 
\label{eq:2ndDerivative}
\end{eqnarray}
 with $\varphi(x+1)$ being value of $\varphi$ at the
position $x+1$ and $h_{x}$ being the step width in $x$ direction. Using the derivatives from \eqref{eq:1stDerivative} and \eqref{eq:2ndDerivative}, \eqref{eq:poisson_cyl} can be rewritten as 
\begin{equation}
\begin{split}
\varphi(r \,\phi \, z) = 
\frac{1}{C} & \left( -\frac{\rho(r \, \phi \, z)}{\epsilon_{0}
\epsilon_{R}} - \frac{1}{r}\frac{\varphi(r+1) - \varphi(r-1)}{2h_{r}} - 
\frac{\varphi(r+1) + \varphi(r-1)}{h_{r}^{2}} \right. \\ 
& \left. - \frac{1}{r^{2}} \frac{\varphi(\phi+1) + \varphi(\phi-1) }{h_{\phi}^{2}} - \frac{\varphi(z+1) + \varphi(z-1)}{h_{z}^{2}} \right ) , 
\end{split}
\label{eq:PotentialByNN}
\end{equation}
with $C = - \frac{2}{h_{r}^{2}} -
\frac{2}{r^{2}h_{\phi}^{2}} - \frac{2}{h_{z}^{2}}  $.
The value of the
potential $\varphi$ at the position $(r,\phi,z)$ is now defined by its
nearest neighours. In such a way the electric potential can be
calculated in an iterative way by defining boundary conditions,
i.e. the applied voltage on the detector surface. By updating the
potential in iteration $n+1$ at each position 
\begin{equation}
\begin{split}
\varphi(r \,\phi \, z)_{n+1}
= \frac{1}{C} & \left( -\frac{\rho(r \, \phi \, z)}{\epsilon_{0}
\epsilon_{R}} \right. - \frac{1}{r}\frac{\varphi(r+1)_{n} -
\varphi(r-1)_{n}}{2h_{r}} - \frac{\varphi(r+1)_{n} + \varphi(r-1)_{n}}{h_{r}^{2}} \\
&  -\left. \frac{1}{r^{2}} \frac{\varphi(\phi+1)_{n} +
\varphi(\phi-1)_{n} }{h_{\phi}^{2}} - \frac{\varphi(z+1)_{n} +
\varphi(z-1)_{n}}{h_{z}^{2}} \right) 
\end{split}
\label{eq:PotentialIteration}
\end{equation}
the potential percolates through the
detector. The update of the potential value at each position needs to
be done several times, until the change of the potential becomes
negligible. This technique is called Gauss-Seidel method. The
convergence of this method can be speed-up considerably. The value of
the potential is then updated in this way 
\begin{equation}
\varphi(r, \phi, z)_{n+1}^{'} =
SORConst \cdot (\varphi(r, \phi, z)_{n+1} - \varphi(r, \phi, z)_{n}) +
\varphi(r, \phi, z)_{n}
\label{eq:SOR}
\end{equation}
 with $1 \le SORConst \le 2$. This is called
Successive Overrelaxation. 
The same method is applied to the weighting potential.

The update of the potential is stopped if the relative overall change
of the potential in one iteration drops below a certain value, which
has to be given by the user, typically $1 \cdot 10^{-5}$.

Having finished updating the potential, differentiation of the potential yields the electric field.


%%% Local Variables:
%%% mode:latex
%%% TeX-master: "GSTR-08-M007"
%%% End:
